\documentclass[letterpaper,12pt]{article}

\usepackage{fancyhdr}
\usepackage[utf8]{inputenc}
\usepackage{microtype}
\usepackage[top=1in, bottom=1in, left=0.75in, right=0.75in]{geometry}
\usepackage{graphicx}
\usepackage{siunitx}
\newcommand{\celsius}{\si{\celsius }}
\newcommand{\kelvin}{\si{\kelvin}}
\newcommand{\sub}[1]{\textsubscript{#1}}
\newcommand{\super}[1]{\textsuperscript{#1}}

\usepackage{hyperref}
\usepackage{amsmath}
\usepackage{booktabs}

% use xelatex to use different font
%\usepackage{fontspec}
\usepackage{mathspec}
%\setallmainfonts(Digits,Latin){Fira Sans Light}
\setmainfont{Fira Sans}% Light}
\setmathrm{Fira Sans}% Light}
\setmathfont(Digits,Latin,Greek){Fira Sans}% Light}
%\usepackage{fontspec}
%\setmainfont{Fira Sans Light}

%input pgf images to use pgf image format from matplotlib 
%\usepackage{pgf}
%\usepackage{unicode-math}
%\setmainfont{Fira Sans Light}
%\setmathfont{Fira Sans Light}
%\setmathfont[range=\mathit]{Fira Sans Light Italic}


\renewcommand{\bigskip}{\vspace{1in}}
\newcommand{\hugeskip}{\vspace{2in}}
\newcommand{\giantskip}{\vspace{3in}}
\newcommand{\blank}{ \rule{1in}{0.1mm} }
\sisetup{retain-explicit-plus}

\newcommand{\university}{Samford University}
\newcommand{\faculty}{Department of Physics}
\newcommand{\class}{General Physics 2}
\newcommand{\examnum}{Worksheet}
\newcommand{\content}{Example Content}
\newcommand{\examdate}{\today}
\newcommand{\timelimit}{}
\newcommand{\myline}{\vspace{.5cm}\hrule}
\newcommand{\week}{}

\pagestyle{fancy}
\setlength\parindent{0in}
\setlength\parskip{0.1in} 
\setlength\headheight{15pt}

\raggedbottom
\widowpenalty=10000
\clubpenalty=10000

%%%%%%%%%%% HEADER / FOOTER %%%%%%%%%%%
\rhead{\textsc{\university\ -- \week}}
\chead{\textsc{}}
\lhead{\textsc{\class}}
\rfoot{\textsc{\thepage}}
\cfoot{\textit{}}
%\lfoot{\textsc{\university}}

%%%%%%%%%%%%%%%%%%%%%%%%%%%%%%%%%%%%%%%

%\pagestyle{headandfoot}
%\firstpageheader{\textsc{\class}}{}{\textsc{\examdate}}
%\firstpageheadrule
%\firstpagefooter{}{}{}
%\runningheader{\textsc{\class}}{\examnum}{\examdate}
%\runningheadrule
%\runningfooter{}{}{\thepage}

\setlength{\parindent}{0pt}
\setlength{\parskip}{\baselineskip}

\title{\class}
\author{\examnum}
\date{\examdate}
\rhead{\textsc{Chapter 3 Homework}}
\renewcommand{\university}{}
\renewcommand{\class}{PHYS330 -- E \& M}
\renewcommand{\week}{}

\begin{document}

\begin{enumerate}
\setlength\itemsep{3 in}

\item
In the notes, we worked an example where the potential along the x-axis is zero and the potential in the x-direction at $y=a$ is also zero, but the potential along $x=0$ from $y=0$ to $y=a$ was a constant $\phi_0$. For this problem, change the potential along the back wall from a single constant, to two constants, so that the potential from $y=0$ to $y=a/2$ is $\phi_0$ and the potential from $y=a/2$ to $y=a$ is $-\phi_0$. Also plot the first several terms of this in Mathematica and do a version of this in Excel with the relaxation method and plot that as well.

\clearpage

\item
For the first example problem (which I also referenced in the previous problem), what would be the surface charge density $\sigma$ of the back plate assuming it was a conductor maintained at a the uniform potential of $\phi_0$?

\clearpage

\item
A rectangular pipe runs along the z-axis. Three of its sides are maintained at $\phi = 0$ (so they are grounded): $y=0$, $y=a$, and $x=0$. The fourth side at $x=b$ is a constant potential $\phi_0$. What is a general expression for the potential inside the pipe? Plot several terms of this in Mathematica and build an Excel model and plot that as well. 

\clearpage

\item
A cubical box with side lengths $a$ has a 5 sides that are grounded, but the top side is maintained at constant potential $\phi_0$. What is a general expression for the potential inside the box?
\clearpage


\end{enumerate}


\end{document}
